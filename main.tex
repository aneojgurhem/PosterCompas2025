\def\ptitle{ArmoniK : An Open-Source Solution for Computation Orchestration and Distribution} % Title
\def\pauthor{Jérôme Gurhem} % Author
\def\pcollaborators{} % Advisors
\def\pteam{Wilfried Kirschenmann} % Team
\def\pinstitute{Aneo, Boulogne-Billancourt, France} % Affiliation
\def\pdate{Jeudi 26 Juin, 2025} % Date
\def\plogo{fig/logo/logo_aneo.png} % Logo of your institution

%\pdfobjcompresslevel=0
\documentclass[final]{beamer}
\usepackage[orientation=portrait,size=a0,scale=1.2]{beamerposter}
\usepackage[utf8]{inputenc}
\usepackage[sfdefault]{roboto}
\usepackage[english]{babel}
\usepackage{amsmath, amsthm, amssymb, array, booktabs, grffile, latexsym, tabularx, xspace, hyperref}
\newcolumntype{Z}{>{\centering\arraybackslash}X}
\newcommand{\pphantom}{\textcolor{ta3aluminium}}
\newlength{\columnheight}

\def\purl{\url{https://2025.compas-conference.fr/}}
\def\pdate{\Large Compas, 26 juin 2025} % Date
\def\pmail{\Large Bordeaux ~~~~~~}

\mode<presentation>{\usetheme{COMPAS}}
\title[\ptitle]{\texorpdfstring{\huge \ptitle}{\ptitle}}
\author[\pauthor]{\pauthor\ -\ \pcollaborators}
\institute[\pinstitute]{\pteam\ -\ \pinstitute}
\date[\pdate]{\pdate}
\setlogo{\plogo}
\setauthorurl{\purl}
\setauthoremail{\pmail}


\usepackage{listings}
\usepackage{tikz}
\usepackage{svg}
\usepackage{relsize}
\usetikzlibrary{positioning, fit, backgrounds, shapes, arrows.meta}

\graphicspath{{./fig/}} % Figures and logos directory
\setlength{\columnheight}{700ex} % Tweak this value if columns are too long/short (should be okay with 588ex)

\definecolor{pykeyword}{rgb}{0.25, 0.3, 0.85}
\definecolor{pycomment}{rgb}{0.0, 0.6, 0.0}
\definecolor{pystring}{rgb}{0.65, 0.1, 0.1}
\definecolor{codebg}{rgb}{0.97, 0.97, 0.97}
\lstdefinestyle{pythonstyle}{
  language=Python,
  basicstyle=\ttfamily\scriptsize,
  keywordstyle=\color{pykeyword}\bfseries,
  commentstyle=\color{pycomment}\itshape,
  stringstyle=\color{pystring},
  backgroundcolor=\color{codebg},
  showstringspaces=false,
  breaklines=true,
  frame=single,
  rulecolor=\color{gray},
  tabsize=2,
  morekeywords={self}, % highlight custom keywords
}

\lstset{style=pythonstyle}



\begin{document}
\begin{frame}[fragile]

  \begin{columns}[T]
    \begin{column}{.49\textwidth}
      \begin{beamercolorbox}[center,wd=\textwidth]{postercolumn}
        \begin{minipage}[T]{.96\textwidth}
            \begin{block}{Context}
              In a world of ever-growing needs for High-Performance Computing (HPC) and massive data processing, \textbf{ArmoniK} provides an Open-Source, scalable platform for executing distributed workloads efficiently on heterogeneous infrastructures.
            \end{block}
        \end{minipage}
      \end{beamercolorbox}
      \vfill
    \end{column}
    \begin{column}{.49\textwidth}
      \begin{beamercolorbox}[center,wd=\textwidth]{postercolumn}
        \begin{minipage}[T]{.96\textwidth}
              \begin{block}{Objectives}
                \begin{itemize}
                  \item Simplify the development and deployment of distributed computing codes
                  \item Maximize resource utilization across private/public clouds and HPC clusters
                  \item Provide a high-level abstraction for developers
                \end{itemize}
              \end{block}
        \end{minipage}
      \end{beamercolorbox}
    \end{column}
  \end{columns}

  \begin{center}
  \begin{minipage}[T]{.975\textwidth}
  \begin{block}{ArmoniK Positionning in HPC}
    \centering
    \includegraphics[width=.8\textwidth]{armonik_framework.pdf}
  \end{block}
  \end{minipage}
  \end{center}

  \begin{columns}[T]
    \begin{column}{.49\textwidth}
      \begin{beamercolorbox}[center,wd=\textwidth]{postercolumn}
        \begin{minipage}[T]{.96\textwidth}
            % \begin{block}{A Serverless Many-Task Computing Platform}
            %     \begin{alertblock}{Serverless}
            %     \begin{quote}
            %         Cloud service category in which the customer can use different cloud capabilities types
            %         without the customer having to provision, deploy and manage either hardware or software
            %         resources, other than providing customer application code or providing customer data.
            %         \\
            %         --- ISO 22123-2:2023
            %     \end{quote}
            %     \end{alertblock}
            %     \begin{alertblock}{Many-Task Computing}
            %     \begin{quote}
            %         Approach that aims to bridge the gap between High-Perfomance Computing and High-Throughput Computing.
            %         \\
            %         --- Wikipedia
            %     \end{quote}
            %     \end{alertblock}         
            % \end{block}

            \begin{block}{Computations/Comm Overlapping}
                \begin{itemize}
                \item ArmoniK is responsible for tasks input and output data management
                \item Allow for automatic communication + scheduling/task execution overlapping
                \item Automatic Uncoordinated Checkpointing
                \end{itemize}
                \vspace{2ex}
                \centering
                \includegraphics[width=0.9\textwidth]{pipelining_compare.png}
            \end{block}

            % \begin{block}{Task Definition in ArmoniK}
            % \lstinputlisting[language=Python]{task-based-programming.py}
            % \end{block}

            \begin{block}{Dynamic Graph}
            \begin{itemize}
                \item Dependency graph is not fully known when scheduling starts
                \item Submissions can happen anytime
                \item Tasks can submit new tasks
                \item Tasks can delegate the production of their output to their new tasks
            \end{itemize}
            \centering
            \vspace{1cm}
            \vfill
            \includesvg[width=0.85\textwidth]{subtasking.svg}
            \end{block}

            \begin{block}{Performance \& Scalability}
            \begin{itemize}
                \item Efficient task retry on failure
                \item Load-aware scheduling
                \item Linear scalability on real workloads
                \item Optimal resource usage on hybrid clusters
                \item Indep : independent tasks workload
                \item Graph : nested fork-join workload
            \end{itemize}
            \begin{figure}
              \centering
              \includegraphics[width=0.8\textwidth]{bench_scalability.pdf}
            \end{figure}
            \end{block}

            % \begin{block}{Dynamic Graph Example}
            % \lstinputlisting[language=Python]{subtasking.py}
            % \end{block}
        \end{minipage}
      \end{beamercolorbox}
    \end{column}
    \begin{column}{.49\textwidth}
      \begin{beamercolorbox}[center,wd=\textwidth]{postercolumn}
        \begin{minipage}[T]{.96\textwidth}
            
            \begin{block}{Simplified Architecture}
            \centering
            \includesvg[width=\textwidth, pretex=\relscale{0.45}]{armonik_architecture.svg}
            \end{block}

            \begin{block}{Main features}
                \begin{itemize}
                \item \textbf{Observability}: GUIs, CLIs, monitoring APIs, metrics, logs, and traces to understand of the state of the system
                \item \textbf{Portability}: Easy to transfer an application from one environment to another
                \begin{itemize}
                    \item Officially supported languages: C\#, C++, Python, Rust, Java, and JavaScript
                    \item Tasks on different architectures (x86, ARM, GPU, Linux, Windows), applications, environments
                \end{itemize}
                \item \textbf{Malleability}: Support dynamic reconfiguration of the number of allocated resources during execution without interruption
                \item \textbf{Resource Sharing}: Allow sharing resources between applications to execute as many as possible at the same
                \item \textbf{Modularity}: Modules can be swapped without modifying ArmoniK's code to suit user neeeds and constraints
                \end{itemize}
            \end{block}

            \begin{block}{Fault Tolerance}
              \begin{itemize}
              \item Works without interruption even when one or more nodes fail
              \item Allow support for preemptible computing resources
              \item Automatic task retry on failure
              \item Each curve represents a percentage of preempted instances
              \end{itemize}
              \begin{figure}
                \centering
                \includesvg[width=0.8\textwidth]{bench_preemption.svg}
              \end{figure}
            \end{block}

            \begin{block}{Conclusion}
              \begin{itemize}
                \item ArmoniK simplifies the development of distributed computing applications.
                \item It ensures efficient execution on clouds and HPC clusters through smart orchestration.
                \item Developers benefit from a high-level abstraction and multi-language SDKs.
                \item Its modular, scalable architecture adapts to changing workloads.
                \item Integrated observability guarantees reliability and performance.
                \item ArmoniK enables the next generation of high-performance, data-intensive computing.
              \end{itemize}
            \end{block}
        \end{minipage}
      \end{beamercolorbox}
    \end{column}
  \end{columns}
  \vskip1ex
\end{frame}
\end{document}